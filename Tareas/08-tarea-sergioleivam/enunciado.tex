\documentclass[letter, 11pt]{article}
%% ================================
%% Packages =======================
\usepackage[utf8]{inputenc}      %%
\usepackage[T1]{fontenc}         %%
\usepackage{lmodern}             %%
\usepackage[spanish]{babel}      %%
\decimalpoint                    %%
\usepackage{fullpage}            %%
\usepackage{fancyhdr}            %%
\usepackage{graphicx}            %%
\usepackage{amsmath}             %%
\usepackage{color}               %%
\usepackage{mdframed}            %%
\usepackage[colorlinks]{hyperref}%%
%% ================================
%% ================================

%% ================================
%% Page size/borders config =======
\setlength{\oddsidemargin}{0in}  %%
\setlength{\evensidemargin}{0in} %%
\setlength{\marginparwidth}{0in} %%
\setlength{\marginparsep}{0in}   %%
\setlength{\voffset}{-0.5in}     %%
\setlength{\hoffset}{0in}        %%
\setlength{\topmargin}{0in}      %%
\setlength{\headheight}{54pt}    %%
\setlength{\headsep}{1em}        %%
\setlength{\textheight}{8.5in}   %%
\setlength{\footskip}{0.5in}     %%
%% ================================
%% ================================

%% =============================================================
%% Headers setup, environments, colors, etc.
%%
%% Header ------------------------------------------------------
\fancypagestyle{firstpage}
{
  \fancyhf{}
  \lhead{\includegraphics[height=4.5em]{LogoDFI.jpg}}
  \rhead{FI3104-1 \semestre\\
         Métodos Numéricos para la Ciencia e Ingeniería\\
         Prof.: \profesor}
  \fancyfoot[C]{\thepage}
}

\pagestyle{plain}
\fancyhf{}
\fancyfoot[C]{\thepage}
%% -------------------------------------------------------------
%% Environments -------------------------------------------------
\newmdenv[
  linecolor=gray,
  fontcolor=gray,
  linewidth=0.2em,
  topline=false,
  bottomline=false,
  rightline=false,
  skipabove=\topsep
  skipbelow=\topsep,
]{ayuda}
%% -------------------------------------------------------------
%% Colors ------------------------------------------------------
\definecolor{gray}{rgb}{0.5, 0.5, 0.5}
%% -------------------------------------------------------------
%% Aliases ------------------------------------------------------
\newcommand{\scipy}{\texttt{scipy}}
%% -------------------------------------------------------------
%% =============================================================


%% =============================================================================
%% CONFIGURACION DEL DOCUMENTO =================================================
%% Llenar con la información pertinente al curso y la tarea
%%
\newcommand{\tareanro}{8}
\newcommand{\fechaentrega}{14/11/2016 23:59 hrs}
\newcommand{\semestre}{2016B}
\newcommand{\profesor}{Valentino González}
%% =============================================================================
%% =============================================================================


\begin{document}
\thispagestyle{firstpage}

\begin{center}
  {\uppercase{\LARGE \bf Tarea \tareanro}}\\
  Fecha de entrega: \fechaentrega
\end{center}


%% =============================================================================
%% ENUNCIADO ===================================================================
\noindent{\large \bf Problema}

Considere un péndulo simple que es forzado periódicamente, de manera que su
movimiento es descrito por la siguiente ecuación:

$$ m L^2 \ddot{\phi} = -mgLsin(\phi) + F_0 cos(\omega t) $$

La frecuencia natural de pequeñas oscilaciones del péndulo es
$\omega_0=\sqrt{g/L}$ pero para oscilaciónes más grandes, la frecuencia es
menor. Se espera, por lo tanto, que el péndulo entre en resonancia para
frecuancias de forzamiento levemente menores que $\omega_0$.

Se pide que Ud. integre numéricamente la ecuación de movimiento y determine
cuál es la frecuencia de forzamiento, $\omega$, para la cual el péndulo alcanza
la máxima amplitiud (después de oscilar muchas veces).

\vspace{0.5em}
\noindent\underline{Indicaciones}

\begin{itemize}
  \item Considere la condición inicial $\phi(0) = \dot{\phi}(0) = 0$.
  \item Use el método de RK4 (debe implementarlo Ud.) con un paso temporal
    $\Delta t$ menor a un centésimo del período natural del péndulo. Explore
    qué valores funcionan mejor para el paso temporal.
  \item Considere los siguientes valores para los parámetros del problema:
    \begin{align*}
      m = 0.85 * 1.0RRR \\
      L = 1.75 * 1.0RRR \\
      F_0 = 0.05 * 1.0RRR
    \end{align*}
    donde $RRR$ son los 3 últimos dígitos de su RUT (no olvide incluirlo en su
    informe).
\end{itemize}

Note que no se trata de integrar la ecuación de movimiento sólo una vez, la
pregunta es para cuál valor de $\omega$ el péndulo alcanza su máxima amplitud.
Debe idear un mecanismo para encontrar ese máximo y describirlo en su informe.


\vspace{1.5em}
\noindent\textbf{Instrucciones Importantes.}
\begin{itemize}

\item \textbf{NO USE JUPYTER NOTEBOOKS}. Empezamos a revisar más en serio el
  diseño del código por lo que es imprescindible que entregue su código en un
  archivo de texto \texttt{.py}.

\item A pesar de que el algoritmo RK4 está implementado en muchas librerías
  libres, en esta tarea debe implementar el algoritmo Ud. En particular,
  \textbf{implemente el algoritmo de modo que sea independiente de la función
  que queremos integrar en este caso en particular}. La idea es que piense en
  cómo hacer su código re-utilizable con el menor esfuerzo posible. Revisaremos
  este aspecto y asignaremos puntaje por cumplir con la condición.

\item También evaluaremos su uso correcto de python. Si define una función
  relativametne larga o con muchos parámetros, recuerde escribir el
  \emph{docstring} que describa los parámetros que recibe la función, el
  output, y el detalle de qué es lo que hace la función. Recuerde que
  generalmente es mejor usar varias funciones cortas (que hagan una sola cosa
  bien) que una muy larga (que lo haga todo).  Utilice nombres explicativos
  tanto para las funciones como para las variables de su código. El mejor
  nombre es aquel que permite entender qué hace la función sin tener que leer
  su implementación ni su \emph{docstring}.

\item Su código debe aprobar la guía sintáctica de estilo
  (\href{https://www.python.org/dev/peps/pep-0008/}{\texttt{PEP8}}). También
  lleva puntaje.

\item Utilice \texttt{git} durante el desarrollo de la tarea para mantener un
  historial de los cambios realizados. La siguiente
  \href{https://education.github.com/git-cheat-sheet-education.pdf}{cheat
    sheet} le puede ser útil. {\bf Revisaremos el uso apropiado de la
  herramienta y asignaremos una fracción del puntaje a este ítem.} Realice
  cambios pequeños y guarde su progreso (a través de \emph{commits})
  regularmente. No guarde código que no corre o compila (si lo hace por algún
  motivo deje un mensaje claro que lo indique). Escriba mensajes claros que
  permitan hacerse una idea de lo que se agregó y/o cambió de un
  \texttt{commit} al siguiente.

\item Para hacer un informe completo Ud. debe decidir qué es interesante y
  agregar las figuras correspondientes. No olvide anotar los ejes e incluir una
  \emph{caption} o título que describa el contenido de cada figura. Tampoco
  olvide las unidades asociadas a las cantidades mostradas en los diferentes
  plots.

\item La tarea se entrega subiendo su trabajo a github. Clone este repositorio
  (el que está en su propia cuenta privada), trabaje en el código y en el
  informe y cuando haya terminado asegúrese de hacer un último \texttt{commit}
  y luego un \texttt{push} para subir todo su trabajo a github.

\item El informe debe ser entregado en formato \texttt{pdf}, este debe ser
  claro sin información de más ni de menos. \textbf{Esto es muy importante, no
  escriba de más, esto no mejorará su nota sino que al contrario}. La presente
  tarea probablemente no requiere informes de más de 3 páginas en total
  (dependiendo de cuántas figuras incluya; esto no es una regla estricta, sólo
  una referencia útil).  Asegúrese de utilizar figuras efectivas y tablas para
  resumir sus resultados. Revise su ortografía.

\end{itemize}

%% FIN ENUNCIADO ===============================================================
%% =============================================================================

\end{document}
