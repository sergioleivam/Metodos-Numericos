\documentclass[letter, 11pt]{article}
%% ================================
%% Packages =======================
\usepackage[utf8]{inputenc}      %%
\usepackage[T1]{fontenc}         %%
\usepackage{lmodern}             %%
\usepackage[spanish]{babel}      %%
\decimalpoint                    %%
\usepackage{fullpage}            %%
\usepackage{fancyhdr}            %%
\usepackage{graphicx}            %%
\usepackage{amsmath}             %%
\usepackage{color}               %%
\usepackage{mdframed}            %%
\usepackage[colorlinks]{hyperref}%%
%% ================================
%% ================================

%% ================================
%% Page size/borders config =======
\setlength{\oddsidemargin}{0in}  %%
\setlength{\evensidemargin}{0in} %%
\setlength{\marginparwidth}{0in} %%
\setlength{\marginparsep}{0in}   %%
\setlength{\voffset}{-0.5in}     %%
\setlength{\hoffset}{0in}        %%
\setlength{\topmargin}{0in}      %%
\setlength{\headheight}{54pt}    %%
\setlength{\headsep}{1em}        %%
\setlength{\textheight}{8.5in}   %%
\setlength{\footskip}{0.5in}     %%
%% ================================
%% ================================

%% =============================================================
%% Headers setup, environments, colors, etc.
%%
%% Header ------------------------------------------------------
\fancypagestyle{firstpage}
{
  \fancyhf{}
  \lhead{\includegraphics[height=4.5em]{LogoDFI.jpg}}
  \rhead{FI3104-1 \semestre\\
         Métodos Numéricos para la Ciencia e Ingeniería\\
         Prof.: \profesor}
  \fancyfoot[C]{\thepage}
}

\pagestyle{plain}
\fancyhf{}
\fancyfoot[C]{\thepage}
%% -------------------------------------------------------------
%% Environments -------------------------------------------------
\newmdenv[
  linecolor=gray,
  fontcolor=gray,
  linewidth=0.2em,
  topline=false,
  bottomline=false,
  rightline=false,
  skipabove=\topsep
  skipbelow=\topsep,
]{ayuda}
%% -------------------------------------------------------------
%% Colors ------------------------------------------------------
\definecolor{gray}{rgb}{0.5, 0.5, 0.5}
%% -------------------------------------------------------------
%% Aliases ------------------------------------------------------
\newcommand{\scipy}{\texttt{scipy}}
%% -------------------------------------------------------------
%% =============================================================
%% =============================================================================
%% CONFIGURACION DEL DOCUMENTO =================================================
%% Llenar con la información pertinente al curso y la tarea
%%
\newcommand{\tareanro}{11}
\newcommand{\fechaentrega}{07/12/2016 23:59 hrs}
\newcommand{\semestre}{2016B}
\newcommand{\profesor}{Valentino González}
%% =============================================================================
%% =============================================================================


\begin{document}
\thispagestyle{firstpage}

\begin{center}
  {\uppercase{\LARGE \bf Tarea \tareanro}}\\
  Fecha de entrega: \fechaentrega
\end{center}


%% =============================================================================
%% ENUNCIADO ===================================================================

\noindent{\large \bf Problema 1}

La ecuación de Fisher-KPP es una llamada ecuación de reacción-difusión que
busca modelar el comportamiento de una especie animal. A continuación se
presenta su versión en 1D:

$$\frac{\partial n}{\partial t} =
\gamma \frac{\partial^2n}{\partial x^2} + \mu n - \mu n^2$$

La variable $n = n(t, x)$ describre la densidad de la especie como función del
tiempo y la posición. Los 3 términos del lado derecho corresponden a:

\begin{itemize}
  \item $\mu n$: la tendencia de la especia a crecer indefinidamente
    (suponiendo que tiene recursos infinitos disponibles).
  \item $-\mu n^2$: Después de un tiempo, el aumento en densidad creará
    competencia por los recursos, lo que tenderá a disminuir la densidad.
  \item $\gamma \nabla^2 n$: La tendencia de la especie a dispersarse para
    encontrar más recursos.
\end{itemize}

La ecuación tiene dos puntos de equilibrio $n=0$ y $n=1$, pero sólo el segundo
es estable. Las soluciones tienen un comportamiento que es una mezcla de
difusión y un pulso viajero.

Para resolver la ecuación discretice la parte de difusión usando el método de
Crank–Nicolson, y el método de Euler explícito para la parte de reacción.
Resuelva la ecuación para x entre 0 y 1 con $\gamma = 0.001$ y $\mu = 1.5$.
Discretice el espacio en aproximadamente 500 puntos y considere las siguientes
condiciones de borde:

\begin{flalign*} 
  n(t, 0) &= 1\\
  n(t, 1) &= 0\\
  n(0, x) &= e^{-x^2/0.1}
\end{flalign*}

Por último, elija su paso temporal de modo que la solución sea estable e
integre hasta al menos t = 4 (en las unidades en que están escritas las
ecuaciónes y las constantes).

Presente la solución encontrada e interprete los resultados.


\vspace{2em}
\noindent{\large \bf Problema 2}

La ecuación de Newell-Whitehead-Segel es otra ecuación de reacción-difusión que
describe fenómenos de convección y combustión entre otros. La ecuación es la
siguiente:

$$\frac{\partial n}{\partial t} =
  \gamma \frac{\partial^2n}{\partial x^2} + \mu ( n - n^3)$$

Esta vez la ecuación tiene 3 puntos de equilibrio $n = 0$ (inestable) y $n =\pm
1$ (estables). Explique en argumentos simples por qué son estables.

Integre esta ecuación siguiendo la misma estrategia que en la pregunta anterior
(mismas constantes también) pero con las siguientes condiciones de borde:

\begin{flalign*}
  n(t, 0) &= 0\\
  n(t, 1) &= 0\\
  n(0, x) &= \texttt{np.random.uniform(low=-0.3, high=0.3, size=Nx)} 
\end{flalign*}

\begin{ayuda}
  NOTA:

  Las condiciones iniciales son aleatorias. Asegúrese de setear las
  condiciones de borde ($n = 0$ para $x=0, 1$) despues de asignar las
  condiciones aleatorias. También es importante setear la semilla al principio
  del script (\texttt{np.random.seed(<algun int>)}), de esa manera su resultado
  será reproducible y no cambiará cada vez que ejecute el script.
\end{ayuda}

Si resolvió la pregunta anterior de manera ordenada y modular, entonces sólo
necesitará hacer un par de pequeños cambios a su código.

Cambie la semilla un par de veces y estudie los cambios en su resultado.

Presente sus resultados mediante los gráficos que le parezcan relevantes e
interprete los resultados.


\vspace{2em}
\noindent\textbf{Instrucciones Importantes.}
\begin{itemize}

\item \textbf{NO USE JUPYTER NOTEBOOKS}. Estamos revisando en serio el diseño
  del código por lo que es imprescindible que entregue su código en un archivo
  de texto \texttt{.py}.

\item Evaluaremos su uso correcto de python. Si define una función
  relativametne larga o con muchos parámetros, recuerde escribir el
  \emph{docstring} que describa los parámetros que recibe la función, el
  output, y el detalle de qué es lo que hace la función. Recuerde que
  generalmente es mejor usar varias funciones cortas (que hagan una sola cosa
  bien) que una muy larga (que lo haga todo).  Utilice nombres explicativos
  tanto para las funciones como para las variables de su código. El mejor
  nombre es aquel que permite entender qué hace la función sin tener que leer
  su implementación ni su \emph{docstring}.

\item Su código debe aprobar la guía sintáctica de estilo
  (\href{https://www.python.org/dev/peps/pep-0008/}{\texttt{PEP8}}). Lleva
  puntaje.

\item Utilice \texttt{git} durante el desarrollo de la tarea para mantener un
  historial de los cambios realizados. La siguiente
  \href{https://education.github.com/git-cheat-sheet-education.pdf}{cheat
    sheet} le puede ser útil. {\bf Revisaremos el uso apropiado de la
  herramienta y asignaremos una fracción del puntaje a este ítem.} Realice
  cambios pequeños y guarde su progreso (a través de \emph{commits})
  regularmente. No guarde código que no corre o compila (si lo hace por algún
  motivo deje un mensaje claro que lo indique). Escriba mensajes claros que
  permitan hacerse una idea de lo que se agregó y/o cambió de un
  \texttt{commit} al siguiente.

\item Para hacer un informe completo Ud. debe decidir qué es interesante y
  agregar las figuras correspondientes. No olvide anotar los ejes e incluir una
  \emph{caption} o título que describa el contenido de cada figura. Tampoco
  olvide las unidades asociadas a las cantidades mostradas en los diferentes
  plots (si es que existen).

\item La tarea se entrega subiendo su trabajo a github. Clone este repositorio
  (el que está en su propia cuenta privada), trabaje en el código y en el
  informe y cuando haya terminado asegúrese de hacer un último \texttt{commit}
  y luego un \texttt{push} para subir todo su trabajo a github.

\item El informe debe ser entregado en formato \texttt{pdf}, este debe ser
  claro sin información de más ni de menos. \textbf{Esto es muy importante, no
  escriba de más, esto no mejorará su nota sino que al contrario}. La presente
  tarea probablemente no requiere informes de más de 3 o 4 páginas en total
  (dependiendo de cuántas figuras incluya; esto no es una regla estricta, sólo
  una referencia útil).  Asegúrese de utilizar figuras efectivas y tablas para
  resumir sus resultados. Revise su ortografía.

\end{itemize}

%% FIN ENUNCIADO ===============================================================
%% =============================================================================

\end{document}
